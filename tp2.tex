\documentclass{homework}

\title{Mathématiques à l'usage des informaticiens\\
TP2: Le code RSA.}
\author{Salmân M'hammed Abibou \& Rodrigo Ferreira Rodrigues}

\begin{document}
	\maketitle
	\onehalfspacing
	\exercise
		\begin{enumerate}
    	\item 
    		\begin{enumerate}
    			\item Pour déchiffrer ce massage Alice va calculer $ M' \equiv C^D \bmod N $
    			\item 
    				\begin{equation*}
    					\begin{split}    				
    				M' &\equiv C^{D} \bmod N \\
    				M' & \equiv 17^7 \bmod 391 \\
    				M' & = 204
    					\end{split}
    				\end{equation*}
   			\end{enumerate}
   		
   		\item Nombres premiers p et q
   		
   	On sait que $p\times q = N$  et $pgcd(p,q) = 1$
		\end{enumerate}




\end{document}


%\item 
%        \begin{enumerate}
%            \item Soit $c = y_1y_2y_3y_4y_5y_6y_7y_8 \in C$ et $N_v$ le nombre de voisins à distance de 2 de c.\\
%        Si on modifie 2 bits on a donc :\\
%            \begin{equation*}
%                N_v = \binom{8}{2} = \frac{8!}{6!\,2!} = 28
%            \end{equation*}
%        Par conséquent on a 28 mots voisins à distance de 2 de c.
%        \item Montrons que $|C| \leq 6$.\\
%        Étant donné que C permet de corriger 2 erreurs, on en déduit que pour 2 mots $c_1$ et $c_2$ de C, aucun voisin à distance au plus de 2 de $c_1$ n'appartient à l'ensemble des voisins à distance au plus de 2 de $c_2$.\\
%        Soit $\beta(c) = \left\{ c' \in \mathbb{F}_2^7 : \delta(c,c')\leq 2 \right\}$\\
%        \begin{equation*}
%        \begin{split}
%            \beta(c) & = \beta_0(c) \cup \beta_1(c) \cup \beta_2(c) \\
%                    & = \left\{c\right\} \cup \left\{ c' \in \mathbb{F}_2^7 : \delta(c,c') = 1 \right\} \cup \left\{ c' \in \mathbb{F}_2^7 : \delta(c,c') = 2\right\} \\
%            |\beta(c)| & = 1 + 8 + 28  && \text{car toutes les boules sont disjointes}\\
%                    & = 37
%        \end{split}
%        \end{equation*}
%         On a donc :
%        \begin{equation*}
%            \begin{split}
%                |C|\times 37 & \leq 256 \\
%                |C| & \leq \frac{256}{37} = 6,92\\
%                |C| & \leq 6
%            \end{split}
%        \end{equation*}
%        avec:
%        \begin{itemize}
%            \item $\beta_0(c)$ l'ensemble des voisins à distance 0 de c c'est-à-dire le mot lui-même.
%            \item $\beta_1(c)$ l'ensemble des voisins à distance 1 de c.
%            \item $\beta_2(c)$ l'ensemble des voisins à distance 2 de c.
%        \end{itemize}
%        
%        \item On suppose que C est linéaire. Soit $A_{(8,k)}$ la matrice associée au code C.\\
%        On sait que $|C|\leq 6$. \\
%        $|C|\leq 6$ \Leftrightarrow $2^k \leq 6$ \Leftrightarrow $k \leq 2,58$ \Leftrightarrow $k = 1 ou k = 2$.
%        \end{enumerate}
%    \item On a: $C:\mathbb{F}_2^2 \to \mathbb{F}_2^8$ car la matrice associée au code est une matrice à $n = 8$ lignes et $k = 2$ colonnes.\\
%    \medskip
%    \newline
%    \begin{equation*}
%        \begin{pmatrix}
%        1 & 0 \\
%        1 & 0 \\
%        1 & 0 \\
%        1 & 1 \\
%        1 & 1 \\
%        0 & 1 \\
%        0 & 1 \\
%        0 & 1 \\
%    \end{pmatrix}
%    \times
%    \begin{pmatrix}
%        x_1 \\
%        x_2 \\
%    \end{pmatrix}
%    =
%    \begin{pmatrix}
%        y_1 = x_1 \\
%        y_2 = x_1 \\
%        y_3 = x_1 \\
%        y_4 = x_1 + x_2 \\
%        y_5 = x_1 + x_2 \\
%        y_6 = x_2 \\
%        y_7 = x_2 \\
%        y_8 = x_2 \\
%    \end{pmatrix}
%    \end{equation*}
%    
%    
%    \medskip
%    \newline
%    Déterminons les mots du code et leur poids
%    \begin{center}
%       \begin{tabular}{|c|c|c|}
%       \hline
%        x & y & w(y) \\
%        \hline
%        00 & 00000000  &   \\
%        01 & 00011111 & 5  \\
%        10 & 11111000 & 5  \\
%        11 & 11100111 & 6  \\
%        \hline
%        \end{tabular} 
%    \end{center}
%    
%    $\delta_{min} = Min(w(y)) = 5$ car C est un code linéaire.\\
%    Les caractéristiques du code C sont: $(n = 8, k = 2, \delta_{min} = 5)$.
