\documentclass{homework}

\title{Mathématiques à l'usage des informaticiens\\
TP2: Le code RSA.}
\author{Salmân M'hammed Abibou \& Rodrigo Ferreira Rodrigues}

\begin{document}
	\maketitle
	\onehalfspacing
	\exercise
		\begin{enumerate}
    	\item 
    		\begin{enumerate}
    			\item Pour déchiffrer ce message Alice va calculer $ M' \equiv C^D \bmod N $
    			\item 
    				\begin{equation*}
    					\begin{split}    				
    				M' &\equiv C^{D} \bmod N \\
    				M' & \equiv 17^7 \bmod 391 \\
    				M' & = 204
    					\end{split}
    				\end{equation*}
   			\end{enumerate}
   		
   		\item Nombres premiers p et q
   		
			\begin{enumerate}
    			\item On sait que $p\times q = N$  et $pgcd(p,q) = 1$. On trouve finanlement que $p = 23$ et $q = 17$.
    			\item 
    				\begin{equation*}
    					\begin{split}    				
    				\varphi(N) & = (p-1) \times (q-1)\\
    				\varphi(N) & = 22 \times 16 \\
    				\varphi(N) & = 352
    					\end{split}
    				\end{equation*}
   			\end{enumerate}   	
   		
   		\item Relation entre $E$ et $D$
   			\begin{equation*}
   				\begin{split}
   					D & = E^{-1} \bmod \varphi(N) \\
   					D & = 151^{-1} \bmod 352 \\
   					D & = 7\text{\footnotemark}
   				\end{split}
   			\end{equation*}
   			\footnotetext[1]{7 étant obtenu grâce à l'algorithme d'Euclide étendu qu'on a programmé au TP1}
		\end{enumerate}

	
	\exercise
		\begin{enumerate}
			\item 
				\begin{enumerate}
    			\item Chiffrement de $M = 112$
    				\begin{equation*}
    					\begin{split}    				
    				C & = M^{E} \bmod N \\
    				C & = 112^{11} \bmod 221 \\
    				C & = 2
    					\end{split}
    				\end{equation*}
    			\item Déchiffrement du cryptogramme $C = 78$
    				\begin{equation*}
    					\begin{split}    				
    				M' & = C^{D} \bmod N \\
    				M' & = 78^{35} \bmod 221 \\
    				M' & = 65
    					\end{split}
    				\end{equation*}
   				\end{enumerate}
   			
   			\item 
				\begin{enumerate}
    			\item Calculs de $N$ et $\varphi(N)$
    			
    				\begin{minipage}{0.4\linewidth}
    					\begin{align*}    					   				    				
    						N & = p \times q \\
    						N & = 53 \times 71 \\
    						N & = 3763
    					\end{align*}
					\end{minipage}\hfill    					
    				\begin{minipage}{0.4\linewidth} 
    					\begin{align*}   				
    						\varphi(N) & = (p-1) \times (q-1)\\
    						\varphi(N) & = 52 \times 10 \\
    						\varphi(N) & = 3640
    					\end{align*}
    				\end{minipage}
    				
    				
    					
    			
    			\item Vérification et calcul de $D$
    			
    			On remarque $E = 307$ < $3640 = \varphi(N)$ et que $pgcd(\varphi(N),E) = pgcd(3640,307) = 1$\footnote[1]{On l'a vérifié grâce au programme qu'on a fait au TP1} donc $E$ est acceptable.
    				\begin{equation*}
    					\begin{split}    				
    				D & = E^{-1} \bmod \varphi(N) \\
   					D & = 307^{-1} \bmod 3640 \\
   					D & = 83
    					\end{split}
    				\end{equation*}
    				
    			\item Elements constitutifs des clés publique et privée
    				\begin{itemize}
    					\item Clé publique $= (E,N) = (307,221)$
    					\item Clé privée $= D = 83$
    				\end{itemize}
    			
    			
    			\item Il faut se débarasser des éléments restants c'est-à-dire de $p$, $q$ et $\varphi(N)$ puisque leur connaissance ne sera plus utile pour la suite du cryptage/décryptage et limite aussi le risque de pirater le code.
   			\end{enumerate}
		\end{enumerate}

	\exercise
		\begin{enumerate}
			\item 
				\begin{enumerate}
					\item Chiffrement du message "METHODE".
						\begin{itemize}
							\item Numériquement, le message correspond à $12;04;19;07;14;03;04.$
							\item Après concaténation on a: $120419071403040.$
							\item Découpe en paquets de 3 : $120;419;071;403;040.$
							\item Chiffrement de chaque paquet : $120^{257} \equiv 589 \bmod 1073$; $419^{257} \equiv 673 \bmod 1073$; $71^{257} \equiv 238 \bmod 1073$; $403^{257} \equiv 308 \bmod 1073$; $40^{257} \equiv 391 \bmod 1073$.
							\item Le cryptogramme est $\mathbf{589;673;238;308;391}$.
						\end{itemize}
					
					\item Déchiffrement du cryptogramme: 263;115;613;10.
						\begin{itemize}
							\item Déchriffrement de chaque paquet : $263^{353} \equiv 21 \bmod 1073$; $115^{353} \equiv 724 \bmod 1073$; $613^{353} \equiv 151 \bmod 1073$; $10^{353} \equiv 914 \bmod 1073$.
							\item Message déchiffré : $021; 724; 151; 914$.
							\item Apres concaténation on a : $021724151914$.
							\item Découpe en paquets de 2 : $02; 17; 24; 15; 19; 14$.
							\item Le message est \textbf{CRYPTO}.
						\end{itemize}
					
					\item Chiffrement du message "AVEZVOUSBIENREUSSI".
						\begin{itemize}
							\item Numériquement, le message correspond à $00;21;04;25;21;14;20;18;01;08;04;13;17;04;20;18;18;08$.
							\item Après concaténation on a: $002104252114201801080413170420181808$.
							\item Découpe en paquets de 3 : $002;104;252;114;201;801;080;413;170;420;181;808$.
							\item Chiffrement de chaque paquet : $2^{257} \equiv 32 \bmod 1073$; $104^{257} \equiv 916 \bmod 1073$; $252^{257} \equiv 546 \bmod 1073$; $114^{257} \equiv 983 \bmod 1073$; $201^{257} \equiv 403 \bmod 1073$; $801^{257} \equiv 1001 \bmod 1073$; $80^{257} \equiv 709 \bmod 1073$; $413^{257} \equiv 857 \bmod 1073$; $170^{257} \equiv 716 \bmod 1073$; $420^{257} \equiv 1034 \bmod 1073$; $181^{257} \equiv 567 \bmod 1073$;$808^{257} \equiv 919 \bmod 1073$.
							\item Le cryptogramme est $\mathbf{32;916;546;983;403;1001;709;857;716;1034;567;919}$.
						\end{itemize}
					
					\item Déchiffrement du cryptogramme: 1019;35;567;36;384;703;99;59.
						\begin{itemize}
							\item Déchriffrement de chaque paquet : $1019^{353} \equiv 180 \bmod 1073$; $35^{353} \equiv 13 \bmod 1073$; $567^{353} \equiv 181 \bmod 1073$; $36^{353} \equiv 517 \bmod 1073$; $384^{353} \equiv 140 \bmod 1073$; $703^{353} \equiv 111 \bmod 1073$; $99^{353} \equiv 041 \bmod 1073$; $59^{353} \equiv 204 \bmod 1073$.
							\item Message déchiffré : $180;013;181;517;140;111;041;204$.
							\item Apres concaténation on a : $180013181517140111041204$.
							\item Découpe en paquets de 2 : $18;00;13;18;15;17;14;01;11;04;12;04$.
							\item Le message est \textbf{SANSPROBLEME}.
						\end{itemize}
						
					\item Déchiffrement du cryptogramme: 553;813.
						\begin{itemize}
							\item Déchriffrement de chaque paquet : $553^{353} \equiv 50 \bmod 1073$; $813^{353} \equiv 813 \bmod 1073$.
							\item Message déchiffré : $050;813$.
							\item Apres concaténation on a : $050813$.
							\item Découpe en paquets de 2 : $05;08;13$.
							\item Le message est \textbf{FIN}.
						\end{itemize}
						
				\end{enumerate}
		\end{enumerate}
\end{document}


%\item 
%        \begin{enumerate}
%            \item Soit $c = y_1y_2y_3y_4y_5y_6y_7y_8 \in C$ et $N_v$ le nombre de voisins à distance de 2 de c.\\
%        Si on modifie 2 bits on a donc :\\
%            \begin{equation*}
%                N_v = \binom{8}{2} = \frac{8!}{6!\,2!} = 28
%            \end{equation*}
%        Par conséquent on a 28 mots voisins à distance de 2 de c.
%        \item Montrons que $|C| \leq 6$.\\
%        Étant donné que C permet de corriger 2 erreurs, on en déduit que pour 2 mots $c_1$ et $c_2$ de C, aucun voisin à distance au plus de 2 de $c_1$ n'appartient à l'ensemble des voisins à distance au plus de 2 de $c_2$.\\
%        Soit $\beta(c) = \left\{ c' \in \mathbb{F}_2^7 : \delta(c,c')\leq 2 \right\}$\\
%        \begin{equation*}
%        \begin{split}
%            \beta(c) & = \beta_0(c) \cup \beta_1(c) \cup \beta_2(c) \\
%                    & = \left\{c\right\} \cup \left\{ c' \in \mathbb{F}_2^7 : \delta(c,c') = 1 \right\} \cup \left\{ c' \in \mathbb{F}_2^7 : \delta(c,c') = 2\right\} \\
%            |\beta(c)| & = 1 + 8 + 28  && \text{car toutes les boules sont disjointes}\\
%                    & = 37
%        \end{split}
%        \end{equation*}
%         On a donc :
%        \begin{equation*}
%            \begin{split}
%                |C|\times 37 & \leq 256 \\
%                |C| & \leq \frac{256}{37} = 6,92\\
%                |C| & \leq 6
%            \end{split}
%        \end{equation*}
%        avec:
%        \begin{itemize}
%            \item $\beta_0(c)$ l'ensemble des voisins à distance 0 de c c'est-à-dire le mot lui-même.
%            \item $\beta_1(c)$ l'ensemble des voisins à distance 1 de c.
%            \item $\beta_2(c)$ l'ensemble des voisins à distance 2 de c.
%        \end{itemize}
%        
%        \item On suppose que C est linéaire. Soit $A_{(8,k)}$ la matrice associée au code C.\\
%        On sait que $|C|\leq 6$. \\
%        $|C|\leq 6$ \Leftrightarrow $2^k \leq 6$ \Leftrightarrow $k \leq 2,58$ \Leftrightarrow $k = 1 ou k = 2$.
%        \end{enumerate}
%    \item On a: $C:\mathbb{F}_2^2 \to \mathbb{F}_2^8$ car la matrice associée au code est une matrice à $n = 8$ lignes et $k = 2$ colonnes.\\
%    \medskip
%    \newline
%    \begin{equation*}
%        \begin{pmatrix}
%        1 & 0 \\
%        1 & 0 \\
%        1 & 0 \\
%        1 & 1 \\
%        1 & 1 \\
%        0 & 1 \\
%        0 & 1 \\
%        0 & 1 \\
%    \end{pmatrix}
%    \times
%    \begin{pmatrix}
%        x_1 \\
%        x_2 \\
%    \end{pmatrix}
%    =
%    \begin{pmatrix}
%        y_1 = x_1 \\
%        y_2 = x_1 \\
%        y_3 = x_1 \\
%        y_4 = x_1 + x_2 \\
%        y_5 = x_1 + x_2 \\
%        y_6 = x_2 \\
%        y_7 = x_2 \\
%        y_8 = x_2 \\
%    \end{pmatrix}
%    \end{equation*}
%    
%    
%    \medskip
%    \newline
%    Déterminons les mots du code et leur poids
%    \begin{center}
%       \begin{tabular}{|c|c|c|}
%       \hline
%        x & y & w(y) \\
%        \hline
%        00 & 00000000  &   \\
%        01 & 00011111 & 5  \\
%        10 & 11111000 & 5  \\
%        11 & 11100111 & 6  \\
%        \hline
%        \end{tabular} 
%    \end{center}
%    
%    $\delta_{min} = Min(w(y)) = 5$ car C est un code linéaire.\\
%    Les caractéristiques du code C sont: $(n = 8, k = 2, \delta_{min} = 5)$.
