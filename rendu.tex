
\documentclass[12pt,french,titlepage]{article}
\usepackage[utf8]{inputenc}
\usepackage{babel}
\usepackage[T1]{fontenc}
\usepackage{mathtools}
\usepackage{amssymb}
\usepackage{amsthm}
\usepackage{amsmath}
\usepackage{hyperref}
\usepackage{graphicx}
\usepackage{float}
\usepackage[dvipsnames]{xcolor}
\definecolor{darkWhite}{rgb}{0.94,0.94,0.94}
\usepackage{tcolorbox,listings}
\lstset{
  aboveskip=3mm,
  belowskip=-2mm,
  backgroundcolor=\color{darkWhite},
  basicstyle=\footnotesize,
  breakatwhitespace=false,
  breaklines=true,
  captionpos=bc,
  commentstyle=\color{ForestGreen},
  deletekeywords={...},
  escapeinside={\%*}{*)},
  extendedchars=true,
  keepspaces=true,
  keywordstyle=\color{blue},
  language=C,
  literate=
  {²}{{\textsuperscript{2}}}1
  {⁴}{{\textsuperscript{4}}}1
  {⁶}{{\textsuperscript{6}}}1
  {⁸}{{\textsuperscript{8}}}1
  {€}{{\euro{}}}1
  {é}{{\'e}}1
  {è}{{\`{e}}}1
  {ê}{{\^{e}}}1
  {ë}{{\¨{e}}}1
  {É}{{\'{E}}}1
  {Ê}{{\^{E}}}1
  {û}{{\^{u}}}1
  {ù}{{\`{u}}}1
  {â}{{\^{a}}}1
  {à}{{\`{a}}}1
  {á}{{\'{a}}}1
  {ã}{{\~{a}}}1
  {Á}{{\'{A}}}1
  {Â}{{\^{A}}}1
  {Ã}{{\~{A}}}1
  {ç}{{\c{c}}}1
  {Ç}{{\c{C}}}1
  {õ}{{\~{o}}}1
  {ó}{{\'{o}}}1
  {ô}{{\^{o}}}1
  {Õ}{{\~{O}}}1
  {Ó}{{\'{O}}}1
  {Ô}{{\^{O}}}1
  {î}{{\^{i}}}1
  {Î}{{\^{I}}}1
  {í}{{\'{i}}}1
  {Í}{{\~{Í}}}1,
  %morekeywords={*,...},
  numbers=left,
  numbersep=10pt,
  numberstyle=\tiny\color{black},
  rulecolor=\color{black},
  showspaces=false,
  showstringspaces=false,
  showtabs=false,
  stepnumber=1,
  stringstyle=\color{gray},
  tabsize=4,
  title=\lstname,
}
\lstdefinestyle{frameStyle}{
    basicstyle=\footnotesize,
    numbers=left,
    numbersep=20pt,
    numberstyle=\tiny\color{black}
}
 
\tcbuselibrary{listings,skins,breakable}
 
\newtcbinputlisting{\cinput}[2][]{
    arc=0mm,
    top=0mm,
    bottom=1mm,
    left=3mm,
    right=0mm,
    width=\textwidth,
    %listing engine=listings,
    listing file={#2},
    listing only,
    listing options={style=frameStyle},
    breakable
}
 

\title{TP Cryptage}
\medskip

\author{Salmân Abibou \& Rodrigo Ferreira Rodrigues \\
Université Clermont Auvergne\\}
\vfill

\date{\today}

\begin{document}
	\maketitle


	\tableofcontents
	\newpage
	
	\section{Rappel des méthodes}
	
	\subsection{PGCD}
	
	Le \textbf{PGCD}, ou  plus grand commun diviseur, est le plus grand entier $d$, diviseur commun de deux entiers, $a$ et $b$.\\
	Il se note PGCD($a,b$).\\
	\\
	\textbf{Exemple : }$14 = 7\times2$ et $21 = 7\times3$ donc PGCD($14,21$)$ = 7$
	
	\subsection{Algorithme d'Euclide Etendu}
	
	Il se repose sur l'\textbf{Identité de Bezout}:\\
	\begin{equation}
	\text{si } d=Pcgd(a,b) \text{ alors } \exists (u,v)\in \mathbb{Z}^2 \text{ tels que }au+bv=d
	\end{equation}
	L'algorithme permet alors de calculer $u$ et $v$.
	\subsection{Code inverse}
	
	Le but est d'inverser un mot : le premier caractère devient dernier, le deuxième devient avant-dernier, ...\\
	Que ce soit pour crypter ou décrypter, le fonctionnement reste inchangé.\\
	\\
	\textbf{Exemple : }\textit{Bonjour} devient \textit{ruojnoB}
	
	\subsection{Code César}
	
	Le but est de chiffrer un mot par décalage des lettres de l'alphabet d'une valeur fixe que l'on appelle clé.\\
	\\
	Si on dépasse la dernière lettre de l'alphabet, on reprend à la première : on travaille alors sur modulo \textit{taille de l'alphabet}.\\
	\\
	On appelle $n$, la taille de l'alphabet.\\
	\\
	On obtient alors la clé de chiffrement suivante :\\
	
	$$
	f:\left\{
	\begin{array}{l}
	{0,..,n} \rightarrow {0,..,n}\\
	x \rightarrow f(x)=x + \textit{clé }\text{mod } n
	\end{array}
	
	\right.
		$$
	\subsection{Code César affine}
\end{document}
