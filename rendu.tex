
\documentclass[12pt,french,titlepage]{article}
\usepackage[utf8]{inputenc}
\usepackage{babel}
\usepackage[T1]{fontenc}
\usepackage{mathtools}
\usepackage{amssymb}
\usepackage{amsthm}
\usepackage{amsmath}
\usepackage{hyperref}
\usepackage{graphicx}
\usepackage{float}
\usepackage[dvipsnames]{xcolor}
\definecolor{darkWhite}{rgb}{0.94,0.94,0.94}
\usepackage{tcolorbox,listings}
\lstset{
  aboveskip=3mm,
  belowskip=-2mm,
  backgroundcolor=\color{darkWhite},
  basicstyle=\footnotesize,
  breakatwhitespace=false,
  breaklines=true,
  captionpos=bc,
  commentstyle=\color{ForestGreen},
  deletekeywords={...},
  escapeinside={\%*}{*)},
  extendedchars=true,
  keepspaces=true,
  keywordstyle=\color{blue},
  language=C,
  literate=
  {²}{{\textsuperscript{2}}}1
  {⁴}{{\textsuperscript{4}}}1
  {⁶}{{\textsuperscript{6}}}1
  {⁸}{{\textsuperscript{8}}}1
  {€}{{\euro{}}}1
  {é}{{\'e}}1
  {è}{{\`{e}}}1
  {ê}{{\^{e}}}1
  {ë}{{\¨{e}}}1
  {É}{{\'{E}}}1
  {Ê}{{\^{E}}}1
  {û}{{\^{u}}}1
  {ù}{{\`{u}}}1
  {â}{{\^{a}}}1
  {à}{{\`{a}}}1
  {á}{{\'{a}}}1
  {ã}{{\~{a}}}1
  {Á}{{\'{A}}}1
  {Â}{{\^{A}}}1
  {Ã}{{\~{A}}}1
  {ç}{{\c{c}}}1
  {Ç}{{\c{C}}}1
  {õ}{{\~{o}}}1
  {ó}{{\'{o}}}1
  {ô}{{\^{o}}}1
  {Õ}{{\~{O}}}1
  {Ó}{{\'{O}}}1
  {Ô}{{\^{O}}}1
  {î}{{\^{i}}}1
  {Î}{{\^{I}}}1
  {í}{{\'{i}}}1
  {Í}{{\~{Í}}}1,
  %morekeywords={*,...},
  numbers=left,
  numbersep=10pt,
  numberstyle=\tiny\color{black},
  rulecolor=\color{black},
  showspaces=false,
  showstringspaces=false,
  showtabs=false,
  stepnumber=1,
  stringstyle=\color{gray},
  tabsize=4,
  title=\lstname,
}
\lstdefinestyle{frameStyle}{
    basicstyle=\footnotesize,
    numbers=left,
    numbersep=20pt,
    numberstyle=\tiny\color{black}
}
 
\tcbuselibrary{listings,skins,breakable}
 
\newtcbinputlisting{\cinput}[2][]{
    arc=0mm,
    top=0mm,
    bottom=1mm,
    left=3mm,
    right=0mm,
    width=\textwidth,
    %listing engine=listings,
    listing file={#2},
    listing only,
    listing options={style=frameStyle},
    breakable
}
 

\title{TP Cryptage}
\medskip

\author{M'hammed Salmân Abibou \& Rodrigo Ferreira Rodrigues \\
Université Clermont Auvergne\\}
\vfill

\date{\today}

\begin{document}
	\maketitle


	\tableofcontents
	\newpage
	
	\section{Rappel des méthodes}
	
	\subsection{PGCD}
	
	Le \textbf{PGCD}, ou  plus grand commun diviseur, est le plus grand entier $d$, diviseur commun de deux entiers, $a$ et $b$.\\
	Il se note PGCD($a,b$).\\
	\\
	\textbf{Exemple : }$14 = 7\times2$ et $21 = 7\times3$ donc PGCD($14,21$)$ = 7$
	
	\subsection{Algorithme d'Euclide Etendu}
	
	Il se repose sur l'\textbf{Identité de Bezout}:\\
	\begin{equation}
	\text{si } d=Pcgd(a,b) \text{ alors } \exists (u,v)\in \mathbb{Z}^2 \text{ tels que }au+bv=d
	\end{equation}
	L'algorithme permet alors de calculer $u$ et $v$.
	\subsection{Code inverse}
	
	Le but est d'inverser un mot : le premier caractère devient dernier, le deuxième devient avant-dernier, ...\\
	Que ce soit pour crypter ou décrypter, le fonctionnement reste inchangé.\\
	\\
	\textbf{Exemple : }\textit{Bonjour} devient \textit{ruojnoB}
	
		Le but est de chiffrer un mot par décalage des caractères de l'alphabet d'une valeur fixe que l'on va ici appeler $b$.\\
	\\
	On appelle $n$, la taille de l'alphabet.\\
	\\
	Si on dépasse la dernier caractère de l'alphabet, on reprend au premier : on travaille alors sur modulo $n$.\\
	\\
	On obtient alors la clé de chiffrement suivante :\\
	
	$$
	f:\left\{
	\begin{array}{l}
	{0,..,n} \rightarrow {0,..,n}\\
	x \rightarrow f(x)=x + b\text{ mod } n
	\end{array}	
	\right.
	$$
	
	\medskip
	
	Posons $y$, le nouveau caractère avec $y=f(x)$. On cherche la fonction $g$ tel que $g(y)=x$.\\
	\\
	On en déduit alors que la clé de déchiffrement est :
	
	$$
	g:\left\{
	\begin{array}{l}
	{0,..,n} \rightarrow {0,..,n}\\
	y \rightarrow g(y)=y - b\text{ mod } n
	\end{array}	
	\right.
	$$
	
	
	\subsection{Code César affine}
	
	Le but est aussi de chiffrer un mot par décalage des caractères de l'alphabet. Or ici, le chiffrement se fait à l'aide d'une fonction affine.\\
	\\
	On obtient une clé de chiffrement de la sorte :\\ 
	\\
	
	$$
	f:\left\{
	\begin{array}{l}
	{0,..,n} \rightarrow {0,..,n}\\
	x \rightarrow f(x)=ax + b\text{ mod } n
	\end{array}	
	\right.
	$$
	
	(avec $a$ et $b$ des entiers et $n$ indiquant toujours la taille de l'alphabet).\\
	\\
	D'ailleurs, pour que le code fonctionne, on doit s'assurer que $PGCD(a,n)=1$, soit que $a$ et $n$ sont premiers entre eux.\\
	\\
	Cependant, dans ce cas me déchiffrement est plus difficile. On doit toujours aboutir à une fonction affine $g$ du type $g(y) =x=a'y+b'$.\\
	\\
	Puisque on a $y=f(x)$ alors on en déduit que :
	\begin{align*}
	g(y)=a'x+b'&=a'(ax+b)+b'[n]\\
	\intertext{d'où}
	&=a'ax+a'b+b'[n]
	\end{align*} 
	\\
	Par identification, on obtient $aa'\equiv1[n]$ et $a'b+b' \equiv 0[n]$.\\
	\\
	$a'$ est donc l'inverse multiplicatif de $a[n]$ et on en déduit que $b\equiv -a'b[26]$.\\
	\\
	On obtient $a'$ grâce à l'algorithme d'Euclide Etendu. En effet, puisque $PGCD(a,n)=1$ alors d'après l'identité de Bezout $\exists(u,v), au+nv=1$, avec ici $u=a'$.
	
	
\section{Programmes python \& Jeux d'essais}	
	
	
	\subsection{Programmes python}
	
		\subsubsection{Code inverse}
		\lstinputlisting[language=python, firstline=4, lastline=12, frame=trBL, caption = Fonction Code Inverse.]{tp1.py}
		
		\subsubsection{Code César simple}
		
		\lstinputlisting[language=python, firstline=20, lastline=44, frame=trBL, caption = Fonction Code César Simple.]{tp1.py}
		
		\subsubsection{PGCG}
		
		\lstinputlisting[language=python, firstline=48, lastline=54, frame=trBL, caption = Fonction PGCD.]{tp1.py}
		
		\subsubsection{Algorithme d'Eucide Etendu}
		
		\lstinputlisting[language=python, firstline=56, lastline=71, frame=trBL, caption = Algorithme d'Euclide Etendu.]{tp1.py}
		
		\subsubsection{Code César Affine}
		
		Dans le code César affine nous allons utilisés les fonctions pgcd et Euclide étendu.
			
		\lstinputlisting[language=python, firstline=74, lastline=101, frame=trBL, caption = Fonction Code César Affine.]{tp1.py}
	
	
	\subsection{Jeux d'essais}	
	
	Pour nos jeux d'essais on a construit un programme principal qui s'opère comme suit :
		\begin{enumerate}
			\item On demande à l'utilisateur quel alphabet il veut utiliser entre les deux qu'on lui propose;
			\item Ensuite, l'utilsateur doit faire un choix du message qu'il veut crypter ou décrypter entre quatre (il saisit un chiffre entre 1 et 4) propositions de messages et a aussi le choix de personnaliser son message en le saisissant lui-même (il saisira un chiffre supérieur à 4, 5 par exemple);
			\item Enfin, il lui est demandé de choisir par quel méthode il veut crypter ou décrypter le message choisi (1 pour le code inverse, 2 pour le code César simple et 3 pour le code César affine).
		\end{enumerate}
		
		
		\subsubsection{Jeu 1}
				
	Crypter/Décrypter : Si deux hommes ont la même opinion. L’un d’eux est de trop


		\subsubsection{Jeu 2}
		
	Décrypter : 'snoçel sed ennod em no uq sruojuot sap emia n ej euq neib erdnerppa a terp
sruojuot sius ej'

		\subsubsection{Jeu 3}
		
	Décrypter : Cyhfibhffnhermertneqreybvaqnafyrcnffrcyhfibhfireermybvaqnafyrshghee (César : cle 13)

		\subsubsection{Jeu 4}
		
	Crypter/Décrypter : ""Chez moi, le secret est enfermé dans une maison aux solides cadenas dont la clé est perdue et la porte scellée."-Les milles et une nuits" (cle 2023)
	
	
	
	
	
	
\end{document}
